%
\section{Track reconstruction}
The (P1+P2) distribution for events with two reconstructed tracks is shown in Figure~\ref{figure:t2_1_smom_0}.
No track selection cuts are applied. A $\gamma \to e^+e^-$ peak  is clearly seen.

\begin{figure}[H]
  \begin{tikzpicture}
    \node[anchor=south west,inner sep=0] at (0,0.) {
      % \node[shift={(0 cm,0.cm)},inner sep=0,rotate={90}] at (0,0) {}
      \makebox[\textwidth][c] {
        \includegraphics[width=0.95\textwidth]{png/pipenu_bpip4b0s54r0100_murat_drpc_ana_t2_1_smom_0}
      }
    };
    % \node [text width=8cm, scale=1.0] at (14.5,0.5) {$\mu_B$, expected background mean};
    % \node [text width=8cm, scale=1.0, rotate={90}] at (1.5,7.5) { $S_{D}$, ``discovery'' signal strength  };
  \end{tikzpicture}
  \caption{
    \label{figure:t2_1_smom_0}
    Sum of the two reconstructed track momenta
  }
  \label{figure:event_display}
\end{figure}

Figure~\ref{figure:t2_1_smom_1_fit} shows the fit of the conversion peak with the function
defined by \ref{}... The offset of the reconstructed peak due to the energy losses is 1 MeV,
and the peak FWHM is also close to 1 MeV.

Comparison of Figure ~\ref{figure:t2_1_smom_1_fit} to Figure ~\ref{figure:t2_1_smom_1_fit}
gives the reconstruction efficiency for preselected events of about 10\% which is fairly low.

\begin{figure}[H]
  \begin{tikzpicture}
    \node[anchor=south west,inner sep=0] at (0,0.) {
      % \node[shift={(0 cm,0.cm)},inner sep=0,rotate={90}] at (0,0) {}
      \makebox[\textwidth][c] {
        \includegraphics[width=0.95\textwidth]{png/pipenu_bpip4b0s54r0100_murat_drpc_ana_t2_1_smom_1_fit}
      }
    };
    % \node [text width=8cm, scale=1.0] at (14.5,0.5) {$\mu_B$, expected background mean};
    % \node [text width=8cm, scale=1.0, rotate={90}] at (1.5,7.5) { $S_{D}$, ``discovery'' signal strength  };
  \end{tikzpicture}
  \caption{
    \label{figure:t2_1_smom_1_fit}
    Sum of the two reconstructed track momenta
  }
  \label{figure:event_display}
\end{figure}

However the Mu2e track reconstruction, first of all - the pattern recognition, has never
been optimized for low momentum tracks. An example of a misreconstructed $\gamma \to e^+e^-$
event is shown in Figure ~\ref{figure:event_display}. In this event there are two particles,
an electron and a positron entering the tracker with the momenta of 83.7 and 43.5 MeV/c
correspondingly. Only one of them, the 83.7 MeV/c electron, has a reconstructed track, and 
the positron track has not been found.  

\begin{figure}[H]
  \begin{tikzpicture}
    \node[anchor=south west,inner sep=0] at (0,0.) {
      % \node[shift={(0 cm,0.cm)},inner sep=0,rotate={90}] at (0,0) {}
      \makebox[\textwidth][c] {
        \includegraphics[width=0.95\textwidth]{png/rpc04b0s54r0100_1210_0_005660}
      }
    };
    % \node [text width=8cm, scale=1.0] at (14.5,0.5) {$\mu_B$, expected background mean};
    % \node [text width=8cm, scale=1.0, rotate={90}] at (1.5,7.5) { $S_{D}$, ``discovery'' signal strength  };
  \end{tikzpicture}
  \caption{
    \label{figure:sum_mom_vd13}
    An example of a 129.4 MeV photon conversion event with two tracks
    in the tracker fiducial and only one track (red circle of a smaller radius)
    reconstructed. Shown are the XY and $\phi$-Z event views in the tracker.
    {\red check if some of the positron hits been lost to the delta-electron removal}
  }
  \label{figure:event_display}
\end{figure}

%%% Local Variables:
%%% mode: latex
%%% TeX-master: t
%%% End:
