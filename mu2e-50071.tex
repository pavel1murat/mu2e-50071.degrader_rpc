% -*- mode:flyspell; mode:latex -*-
\documentclass[12pt]{article}
\addtolength{\oddsidemargin} {-0.885in}
\addtolength{\textwidth}{1.75in}
\addtolength{\evensidemargin}{-0.8in}


\usepackage[latin1]{inputenc}
\usepackage[T1]{fontenc}
\usepackage[english]{babel}
\usepackage{graphicx}
\usepackage{float}
%% \usepackage{siunitx}

%% \usepackage{gensymb}


\usepackage{tikz}
\usepackage{[caption}
\usetikzlibrary{arrows}
\usetikzlibrary{decorations.markings}
\usetikzlibrary{decorations.pathmorphing}
% \usepackage[absolute,overlay]{textpos}
% \usepackage{onimage}

\usepackage{tabularx}
\usepackage{times}
\usepackage{graphics}

% \usepackage{subfigure}
% \usepackage{scalefnt}
%
% \renewcommand\thesubfigure{\arabic{subfigure}}

\usepackage{amsmath}
\usepackage{hyperref}
\usepackage{hhline}
\usepackage{subfig}
\usepackage{color}
\usepackage[all]{hypcap}

\usepackage[normalem]{ulem}  % for striking out
% \usepackage{fancyhdr}
% \pagestyle{fancy}
% \fancyhead[C]{}
% \fancyhead[L] {\it{Mu2e-doc-29670-v1.0} }
%%%%%%%%%%%%%%%%%%%%%%%%%%%%%%%%%%%%%%%%%%%%%%%%%%%%%%%%%%%%%%%%%%%%%%%%%%%%%%
% use natbib - biblatex not available on Mu2e interactive nodes
%%%%%%%%%%%%%%%%%%%%%%%%%%%%%%%%%%%%%%%%%%%%%%%%%%%%%%%%%%%%%%%%%%%%%%%%%%%%%%
\usepackage[square,sort,comma,numbers]{natbib}

% location of the .bib files: env var BIBINPUTS (~/library/bibliography)

% \usepackage[backend=biber, style=numeric-comp, sorting=ynt] {biblatex}
% \addbibresource{clfv.bib}

% \addbibresource{stntuple.bib}
% \addbibresource{mu2e_web.bib}
% \addbibresource{radiative_pion_capture.bib}

\graphicspath{{figures/}}
%%%%%%%%%%%%%%%%%%%%%%%%%%%%%%%%%%%%%%%%%%%%%%%%%%%%%%%%%%%%%%%%%%%%%%%%%%%%%%
% for portability, make sure all commands are included locally,
%%%%%%%%%%%%%%%%%%%%%%%%%%%%%%%%%%%%%%%%%%%%%%%%%%%%%%%%%%%%%%%%%%%%%%%%%%%%%%
\definecolor{ForestGreen}{RGB}{30,139,30}
%\include{commands}
\newcommand {\blue}      {\color{blue}}
\newcommand {\green}     {\color{ForestGreen}}
\newcommand {\red}       {\color{red}}
\newcommand {\purple}    {\color{purple}}
\newcommand {\violet}    {\color{violet}}

\newcommand {\kmax}      {\mbox{$k_{\rm max}$}}
\newcommand {\piplusenu} {\mbox{$\pi^+ \to e^+ \nu$}}

\newcommand {\mumemconv}[1][A] {\mbox{$\mu^- \textrm{#1} \rightarrow e^- \textrm{#1}$}}
% Define a relay to have 2 default arguments instead of limit of 1
\newcommand {\mumepconv}[1][A] {%
  \def\ArgI{{#1}}%store the first argument
  \mumepconvRelay
}
\newcommand \mumepconvRelay[1][A]  {\mbox{$\mu^- \textrm{\ArgI} \rightarrow e^+ \textrm{#1}$}}
\newcommand {\MuToEm}     {\mbox{$\mu^- \ra e^-$}}
\newcommand {\MuToEp}     {\mbox{$\mu^- \ra e^+$}}
\newcommand {\MuPToEp}    {\mbox{$\mu^+ \ra e^+$}}
\newcommand {\ra}        {\rightarrow}
\newcommand {\Rmue}       {\mbox{$R_{\mu e}$}}
\newcommand {\tandip}    {\mbox{$\tan \lambda$}}

\newcommand {\Pb}[1]     {\mbox{$\rm ^{#1}Pb$}}                 % isotopes of lead
\newcommand {\Au}[1]     {\mbox{$\rm ^{#1}Au$}}                 % isotopes of gold
\newcommand {\Ir}[1]     {\mbox{$\rm ^{#1}Ir$}}                 % isotopes of iridium
%%%%%%%%%%%%%%%%%%%%%%%%%%%%%%%%%%%%%%%%%%%%%%%%%%%%%%%%%%%%%%%%%%%%%%%%%%%%%%
% editing commands
%%%%%%%%%%%%%%%%%%%%%%%%%%%%%%%%%%%%%%%%%%%%%%%%%%%%%%%%%%%%%%%%%%%%%%%%%%%%%%
\newcommand {\del}[1]    {{\blue \sout{#1}}}
\newcommand {\dlt}[1]    {{\violet \sout{#1}}} %alternate delete color
\newcommand {\add}[1]    {{\red #1}}
\newcommand {\alt}[1]    {{\green #1}} %alternate comment color
%%%%%%%%%%%%%%%%%%%%%%%%%%%%%%%%%%%%%%%%%%%%%%%%%%%%%%%%%%%%%%%%%%%%%%%%%%%%%%
\begin{document}

\begin{titlepage}
  \begin{flushright}
    \bf {MU2E/PHYSICS/50071} \\
    version 1.01
    \today
 \end{flushright}

  \vspace{1cm}

  \begin{center}
    {\Large \bf On a possibility of the Mu2e momentum scale calibration at full field
      \vspace{0.3in}
      1. Initial analysis
    }

    \vspace{1cm}
    K.Ciampa(BU), P.Murat(FNAL)

    % \footnote{\texttt{Fermilab; e-mail: murat@fnal.gov}}
    \vspace{0.3cm}

    \vspace{0.8cm}
  \end{center}

  \begin{abstract}
    \vspace{0.2in}
    Radiative pion capture (RPC) on hydrogen  provides a unique opportunity to calibrate
    the Mu2e momentum scale at full field.

    Kinematic edge of the momentum distribution of 129.4 MeV RPC photons
    reconstructed in $\gamma \to e^+e^-$ channel could be used as a calibration line.

    We investigate whether the rate of events of interest is sufficient for the calibration.
  \end{abstract}

\end{titlepage}
% \frontmatter
% \chapter*{Abstract}
%
% \addcontentsline{toc}{chapter}{Abstract}
%
% \mainmatter
%
{\tableofcontents}

%%%%%%%%%%%%%%%%%%%%%%%%%%%%%%%%%%%%%%%%%%%%%%%%%%%%%%%%%%%%%%%%%%%%%%%%%%%%%%%
%\chapter{Calibration}
%%%%%%%%%%%%%%%%%%%%%%%%%%%%%%%%%%%%%%%%%%%%%%%%%%%%%%%%%%%%%%%%%%%%%%%%%%%%%%%
% \input{input_data}

%%%%%%%%%%%%%%%%%%%%%%%%%%%%%%%%%%%%%%%%%%%%%%%%%%%%%%%%%%%%%%%%%%%%%%%%%%%%%%%

\newpage
\section {Revision History and TODO items}

\begin{itemize}
\item
  v1.01: inital version
\end{itemize}

TODO items:

\begin{itemize}
\item
  placeholder
\end{itemize}

%%%%%%%%%%%%%%%%%%%%%%%%%%%%%%%%%%%%%%%%%%%%%%%%%%%%%%%%%%%%%%%%%%%%%%%%%%%%%%
\newpage
\section {Introduction}


%%%%%%%%%%%%%%%%%%%%%%%%%%%%%%%%%%%%%%%%%%%%%%%%%%%%%%%%%%%%%%%%%%%%%%%%%%%%%%
\section{RPC on hydrogen}

Has been measured in \cite{RPC_1972_Bistirlich_PhysRevC.5.1867}.

Events of interest for calibration are the monochomatic RPC photons at 129.4 MeV from $\pi^{-} p \to n \gamma$ on
hydrogen. Measurement of this process is shown in 
Figure~\ref{figure:1971_bistirlich_fig_02_h2}
, the photon energy spectrum from pion capture on hydrogen, which also includes lower energy photons from the charge exchange reaction. 
For the $\rm{CH_2}$ spectrum in 
Figure~\ref{figure:1971_bistirlich_fig_09_ch2}
, photons from RPC on carbon occupy the region between the two hydrogen peaks.
To produce a photon in the 129.4 MeV peak, a negative pion stopped on
a $\rm{CH_2}$  degrader must both (i) capture on a hydrogen rather than carbon nucleus, and (ii) interact
radiatively with the proton via $\pi^{-} p \to n \gamma$ rather than producing an intermediate $\pi^0$.


\begin{figure}[H]
 \begin{minipage}{.5\textwidth}
  \includegraphics[width=0.9\textwidth]{png/1971_bistirlich_fig_02_h2}
  \captionsetup{width=.8\linewidth}
  \caption[width=0.9\textwidth]{
      \label{figure:1971_bistirlich_fig_02_h2}
    \kate{Photon energy spectrum from negative pion capture on hydrogen \cite{RPC_1972_Bistirlich_PhysRevC.5.1867}.}
    }
 \end{minipage}
 \begin{minipage}{.5\textwidth}
  \includegraphics[width=0.9\textwidth]{png/1971_bistirlich_fig_09_ch2}
  \captionsetup{width=.8\linewidth}
  \caption[width=0.9\textwidth]{
  \label{figure:1971_bistirlich_fig_09_ch2}
    \kate{Photon energy spectrum from negative pion capture on $\rm{CH_2}$ \cite{RPC_1972_Bistirlich_PhysRevC.5.1867}.}
   }
 \end{minipage}
\end{figure}

The probability of pion capture on the hydrogen component of the $\rm{CH_2}$ compound is
$\rm{ W_H = (12.9 \pm 1.8) \times 10^{-3 }}$
 as used in Ref.~X \path{1991_Harston_PhysRevA.44.103}
 , an average of previous measurements. 
 This accounts for both hydrogen atoms in the $\rm{CH_2}$ molecule, and is larger than but consistent with the
 corresponding measurement (expressed as percent) in Ref.~\cite{RPC_1972_Bistirlich_PhysRevC.5.1867}.
 Negative pions captured on hydrogen have a $ (41.4 \pm 3.2) \% $
 probability of interacting via $\pi^{-} p \to n \gamma$ and producing a photon at 129.4 MeV
 \cite{RPC_1972_Bistirlich_PhysRevC.5.1867}.
 The remaining pions captured on hydrogen produce two photons via the charge exchange reaction
 $ \pi^{-} p \to n \pi^0 $ with $\pi^0  \to \gamma \gamma $,
 which results in the lower energy peak of the hydrogen spectrum. 

Together these factors yield a probability of $ (5.34 \pm 0.85) \times 10^{-3} $
for obtaining 129.4 MeV RPC photons from pion capture on $\rm{CH_2}$.

% \bigskip
%  Additional reference for \path{radiative_pion_capture.bib}:
% {\tiny
% {\begin{verbatim}
% @article{1991_Harston_PhysRevA.44.103,
%   title = {Capture and transfer of stopped pions in alcohols},
%   author = {Harston, M. R. and Armstrong, D. S. and Measday, D. F. and Stanislaus, S. and Weber, P. and Horv\'ath, D.},
%   journal = {Phys. Rev. A},
%   volume = {44},
%   issue = {1},
%   pages = {103--110},
%   numpages = {0},
%   year = {1991},
%   month = {Jul},
%   publisher = {American Physical Society},
%   doi = {10.1103/PhysRevA.44.103},
%   url = {https://link.aps.org/doi/10.1103/PhysRevA.44.103}
% }
% \end{verbatim}}}
% 

\kate{
\bigskip
  Potential changes:
  \begin{itemize}
  \item 
    Include individual previous measurements if needed.
  \end{itemize}
}


%%% Local Variables:
%%% mode: latex
%%% TeX-master: t
%%% End:

% %%%%%%%%%%%%%%%%%%%%%%%%%%%%%%%%%%%%%%%%%%%%%%%%%%%%%%%%%%%%%%%%%%%%%%%%%%%%%%
\section{Estimate of the time needed for momentum calibration}
Calibration with RPC on hydrogen allows to determine the momentum scale of the experiment.

In this section we assume that the experimental background doesn't affect
the determination accuracy of the momentum edge of RPC photon peak
and estimate the time needed for the calibration.

\kate{
\begin{itemize}
\item 
From $ 2.5 \times 10^8 $ protons on target, 67.42 negative pions stop on the $\rm{CH_2}$ part of the degrader at T>200ns.
\item 
A negative pion stopped in $\rm{CH_2}$ has a probability of about $ 5 \times 10^{-3} $ of producing a 129.4 MeV RPC photon from capture on hydrogen (Section 3).
\item 
The optimal geometry from Section 4 places a gold converter at radius R=25.0cm with thickness 0.1mm. Of $10^5$ RPC photons generated with angle 0<$\cos \theta$<0.1, 139 produce $e^+ e^-$ pairs that can likely be reconstructed in the tracker. Events counted as likely reconstructable have $e^+$ and $e^-$ each with at least 20 straw hits and momentum P > 30MeV/c. 
\end{itemize}
The product of these first three factors is a calibration event rate on the order of $10^{-12} $ per POT.
}

\begin{itemize}
\item 
  required accuracy of the momentum scale calibration $@$100 MeV/c: $\sigma_P/P$ < 100 keV/c
\item
  {\red how many events do we need ?} \kate{Estimates in the remaining items assume that 1000 $ \gamma \to e^+ e^- $ events would be sufficient to reconstruct the high-energy edge of the 129.4 MeV RPC photon momentum distribution.}
\item
  \kate{For a yield of $10^{-12}$ events/POT, collecting 1000 events requires $10^{15}$ protons on target.
  In one-batch mode, an average expected pulse intensity is $1.6 \times 10^7$, with one pulse every 1695ns for about 0.4s of each 1.4s Main Injector cycle. The resulting rate is $2.7 \times 10^{12}$ protons/sec. }
%\item
%  For a yield of $10^{-13}$ events/POT, collecting 1000 events requires $10^{16}$ protons on target.
%  In one-batch mode, an average expected pulse intensity is $1.6 \times 10^7$, and
%  an average  pulse rate of 1$.6 \times 10^5$ pulses/sec correspond to the rate of $2.5 \times 10^{12}$ protons/sec.
\item
  Assuming running at 10\% of nominal beam intensity and the data collection efficiency of 50\%,
  collecting 1000 reconstructable \kate{ $ \gamma \to e^+ e^- $ events would require on the order of
  $10^{15}/( 0.1 \cdot 0.5 \cdot  2.7 \times 10^{12}) \sim 10^4$ seconds, achievable in} one day of running.
%\item
%  Assuming running at 10\% of nominal beam intensity and the data collection efficiency of 50\%,
%  collecting 1000 reconstructable \piplusenu\ events would require
%  $10^{16}/(1.25 \times 10^{11}) \sim 10^5$ seconds, or about one day of running.
\item
  running at 10\% of the nominal beam intensity in one-batch mode and with the digitization starting
  at 200 ns corresponds to the total number of background hits per microbunch of about 200,
  so the pileup at T>300 ns should not be a problem.
\end{itemize}

\kate{
\bigskip
  Potential changes:
  \begin{itemize}
  \item 
    Update with simulation results using larger photon angle. Earlier results here follow Section 4 for consistency. 
  \end{itemize}
}
%%% Local Variables:
%%% mode: latex
%%% TeX-master: t
%%% End:


%%%%%%%%%%%%%%%%%%%%%%%%%%%%%%%%%%%%%%%%%%%%%%%%%%%%%%%%%%%%%%%%%%%%%%%%%%%%%% 
\section {Summary}

to be written
%%%%%%%%%%%%%%%%%%%%%%%%%%%%%%%%%%%%%%%%%%%%%%%%%%%%%%%%%%%%%%%%%%%%%%%%%%%%%%
%
%%%%%%%%%%%%%%%%%%%%%%%%%%%%%%%%%%%%%%%%%%%%%%%%%%%%%%%%%%%%%%%%%%%%%%%%%%%%%%
\newpage
\bibliographystyle{unsrtnat}
\bibliography{clfv,mu2e_internal_notes,mu2e_piplusenu_notes}

% \include{appendix_a}
% \appendix

%%%%%%%%%%%%%%%%%%%%%%%%%%%%%%%%%%%%%%%%%%%%%%%%%%%%%%%%%%%%%%%%%%%%%%%%%%%%%%
\section {Datasets}
\label{appendix_b}

\begin{itemize}
\item 
  Definition of the datasets used for this study and the book-keeping information
  can be found at \\
  \href{https://github.com/sridhar130/pipenu/blob/main/doc/datasets.org}
  {\blue https://github.com/sridhar130/pipenu/blob/main/doc/datasets.org}.
\item
  the datasets and stntuples are available from {\bf mu2egpvm*:/exp/mu2e/data/projects/pipenu}
\item
  location of the stntuple catalogs : \\
  \href{https://mu2e.fnal.gov/public/hep/computing/Stntuple/cafdfc/pipenu/index.shtml}
  {\blue https://mu2e.fnal.gov/public/hep/computing/Stntuple/cafdfc/pipenu/index.shtml}
\end{itemize}

%%% Local Variables:
%%% mode: latex
%%% TeX-master: t
%%% End:


\end{document}
