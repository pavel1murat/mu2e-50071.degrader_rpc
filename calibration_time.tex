%%%%%%%%%%%%%%%%%%%%%%%%%%%%%%%%%%%%%%%%%%%%%%%%%%%%%%%%%%%%%%%%%%%%%%%%%%%%%%
\section{Estimate of the time needed for momentum calibration}
Calibration with RPC on hydrogen allows to determine the momentum scale of the experiment.

In this section we assume that the experimental background doesn't affect
the determination accuracy of the momentum edge of RPC photon peak
and estimate the time needed for the calibration.

\kate{
\begin{itemize}
\item 
From $ 2.5 \times 10^8 $ protons on target, 67.42 negative pions stop on the $\rm{CH_2}$ part of the degrader at T>200ns.
\item 
A negative pion stopped in $\rm{CH_2}$ has a probability of about $ 5 \times 10^{-3} $ of producing a 129.4 MeV RPC photon from capture on hydrogen (Section 3).
\item 
The optimal geometry from Section 4 places a gold converter at radius R=25.0cm with thickness 0.1mm. Of $10^5$ RPC photons generated with angle 0<$\cos \theta$<0.2, [139 $\rightarrow$ replace with value from simulation result with updated angle] produce $e^+ e^-$ pairs that can likely be reconstructed in the tracker. Events counted as likely reconstructable have $e^+$ and $e^-$ each with at least 20 straw hits and momentum p>=40MeV. 
Kate $\rightarrow$Will check the results of the new simulation with larger angle, and also check conditions.
\end{itemize}
The product of these first three factors is a calibration event rate on the order of $10^{-12} $ per POT.
}

\begin{itemize}
\item 
  required accuracy of the momentum scale calibration $@$100 MeV/c: $\sigma_P/P$ < 100 keV/c
\item
  {\red how many events do we need ?}
\item
  For a yield of $10^{-13}$ events/POT, collecting 1000 events requires $10^{16}$ protons on target.
  In one-batch mode, an average expected pulse intensity is $1.6 \times 10^7$, and
  an average  pulse rate of 1$.6 \times 10^5$ pulses/sec correspond to the rate of $2.5 \times 10^{12}$ protons/sec.
\item
  Assuming running at 10\% of nominal beam intensity and the data collection efficiency of 50\%,
  collecting 1000 reconstructable \piplusenu\ events would require
  $10^{16}/(1.25 \times 10^{11}) \sim 10^5$ seconds, or about one day of running.
\item
  running at 10\% of the nominal beam intensity in one-batch mode and with the digitization starting
  at 200 ns corresponds to the total number of background hits per microbunch of about 200,
  so the pileup at T>300 ns should not be a problem.
\end{itemize}

%%% Local Variables:
%%% mode: latex
%%% TeX-master: t
%%% End:
