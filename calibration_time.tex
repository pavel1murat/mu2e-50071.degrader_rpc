
%%%%%%%%%%%%%%%%%%%%%%%%%%%%%%%%%%%%%%%%%%%%%%%%%%%%%%%%%%%%%%%%%%%%%%%%%%%%%%
\section{Estimate of the time needed for momentum calibration}
Calibration with RPC on hydrogen allows to determine the momentum scale of the experiment.

In this section we assume that the experimental background doesn't affect
the determination accuracy of the momentum edge of RPC photon peak
and estimate the time needed for the calibration.

\begin{itemize}
\item 
From $ 2.5 \times 10^8 $ protons on target, 67.42 negative pions stop on the $\rm{CH_2}$ part of the degrader at T>200ns.
\item 
  A negative pion stopped in $\rm{CH_2}$ has a probability of about $ 5 \times 10^{-3} $ of producing a 129.4 MeV RPC photon from capture on hydrogen (Section 3).
\item 
The optimal geometry from Section 4 places a gold converter at radius R=25.0cm with thickness 0.1mm. Of $10^7$ RPC photons generated with angle 0<$\cos \theta$<0.2, 10789 produce $e^+ e^-$ pairs that can likely be reconstructed in the tracker (Section 5). Events counted as likely reconstructable have $e^+$ and $e^-$ each with at least 20 straw hits and momentum P > 30MeV/c. Scaling to the full range of photon angles reduces the yield by a factor of 0.2/2, to around $10^{-4}$  per RPC photon. 
\end{itemize}
The product of these first three factors is a calibration event rate on the order of $10^{-13} $ per POT.

\begin{itemize}
\item 
  required accuracy of the momentum scale calibration $@$100 MeV/c: $\sigma_P/P$ < 100 keV/c
\item
Estimates in the remaining items assume that reconstructed tracks from 1000 $ \gamma \to e^+ e^- $ events would be sufficient to reconstruct the high-energy edge of the 129.4 MeV RPC photon momentum distribution. Examples are the fitted peaks in Sections 6-7, which both have about 1000 events in the tracker using the default reconstruction.  
\item
For a yield of $10^{-13}$ events/POT, collecting 1000 events requires $10^{16}$ protons on target.
  In one-batch mode, an average expected pulse intensity is $1.6 \times 10^7$, with one pulse every 1695ns for about 0.4s of each 1.4s Main Injector cycle. The resulting rate is $2.7 \times 10^{12}$ protons/sec, which would provide 1000 calibration events in less than an hour.
\item
  Assuming that calibration data is taken at 10\% of nominal beam intensity with a data collection efficiency of 50\%, the time needed increases by a factor of 20. A further reconstruction efficiency factor of 50\% takes into account events not reconstructed in the tracker, an estimate for the case after improving reconstruction efficiency. Overall, collecting 1000 reconstructable $ \gamma \to e^+ e^- $ events would require on the order of $10^5$ seconds, or about one day of running.
\item
  running at 10\% of the nominal beam intensity in one-batch mode and with the digitization starting
  at 200 ns corresponds to the total number of background hits per microbunch of about 200,
  so the pileup at T>300 ns should not be a problem.
\end{itemize}


%%% Local Variables:
%%% mode: latex
%%% TeX-master: t
%%% End:
