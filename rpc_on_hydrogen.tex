
%%%%%%%%%%%%%%%%%%%%%%%%%%%%%%%%%%%%%%%%%%%%%%%%%%%%%%%%%%%%%%%%%%%%%%%%%%%%%%
\section{RPC on hydrogen}

Has been measured in \cite{RPC_1972_Bistirlich_PhysRevC.5.1867}.

Events of interest for calibration are the monochomatic RPC photons at 129.4 MeV from $\pi^{-} p \to n \gamma$ on
hydrogen. Measurement of this process is shown in 
Figure~\ref{figure:1971_bistirlich_fig_02_h2}
, the photon energy spectrum from pion capture on hydrogen, which also includes lower energy photons from the charge exchange reaction. 
For the $\rm{CH_2}$ spectrum in 
Figure~\ref{figure:1971_bistirlich_fig_09_ch2}
, photons from RPC on carbon occupy the region between the two hydrogen peaks.
To produce a photon in the 129.4 MeV peak, a negative pion stopped on
a $\rm{CH_2}$  degrader must both (i) capture on a hydrogen rather than carbon nucleus, and (ii) interact
radiatively with the proton via $\pi^{-} p \to n \gamma$ rather than producing an intermediate $\pi^0$.


\begin{figure}[H]
 \begin{minipage}{.5\textwidth}
  \includegraphics[width=0.9\textwidth]{png/1971_bistirlich_fig_02_h2}
  \captionsetup{width=.8\linewidth}
  \caption[width=0.9\textwidth]{
      \label{figure:1971_bistirlich_fig_02_h2}
    \kate{Photon energy spectrum from negative pion capture on hydrogen \cite{RPC_1972_Bistirlich_PhysRevC.5.1867}.}
    }
 \end{minipage}
 \begin{minipage}{.5\textwidth}
  \includegraphics[width=0.9\textwidth]{png/1971_bistirlich_fig_09_ch2}
  \captionsetup{width=.8\linewidth}
  \caption[width=0.9\textwidth]{
  \label{figure:1971_bistirlich_fig_09_ch2}
    \kate{Photon energy spectrum from negative pion capture on $\rm{CH_2}$ \cite{RPC_1972_Bistirlich_PhysRevC.5.1867}.}
   }
 \end{minipage}
\end{figure}

The probability of pion capture on the hydrogen component of the $\rm{CH_2}$ compound is
$\rm{ W_H = (12.9 \pm 1.8) \times 10^{-3 }}$
 as used in Ref.~X \path{1991_Harston_PhysRevA.44.103}
 , an average of previous measurements. 
 This accounts for both hydrogen atoms in the $\rm{CH_2}$ molecule, and is larger than but consistent with the
 corresponding measurement (expressed as percent) in Ref.~\cite{RPC_1972_Bistirlich_PhysRevC.5.1867}.
 Negative pions captured on hydrogen have a $ (41.4 \pm 3.2) \% $
 probability of interacting via $\pi^{-} p \to n \gamma$ and producing a photon at 129.4 MeV
 \cite{RPC_1972_Bistirlich_PhysRevC.5.1867}.
 The remaining pions captured on hydrogen produce two photons via the charge exchange reaction
 $ \pi^{-} p \to n \pi^0 $ with $\pi^0  \to \gamma \gamma $,
 which results in the lower energy peak of the hydrogen spectrum. 

Together these factors yield a probability of $ (5.34 \pm 0.85) \times 10^{-3} $
for obtaining 129.4 MeV RPC photons from pion capture on $\rm{CH_2}$.

% \bigskip
%  Additional reference for \path{radiative_pion_capture.bib}:
% {\tiny
% {\begin{verbatim}
% @article{1991_Harston_PhysRevA.44.103,
%   title = {Capture and transfer of stopped pions in alcohols},
%   author = {Harston, M. R. and Armstrong, D. S. and Measday, D. F. and Stanislaus, S. and Weber, P. and Horv\'ath, D.},
%   journal = {Phys. Rev. A},
%   volume = {44},
%   issue = {1},
%   pages = {103--110},
%   numpages = {0},
%   year = {1991},
%   month = {Jul},
%   publisher = {American Physical Society},
%   doi = {10.1103/PhysRevA.44.103},
%   url = {https://link.aps.org/doi/10.1103/PhysRevA.44.103}
% }
% \end{verbatim}}}
% 

\kate{
\bigskip
  Potential changes:
  \begin{itemize}
  \item 
    Include individual previous measurements if needed.
  \end{itemize}
}


%%% Local Variables:
%%% mode: latex
%%% TeX-master: t
%%% End:
